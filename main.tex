\documentclass[12pt]{article}
\usepackage{amsmath, amssymb, amsthm}
\usepackage{enumitem}
\usepackage[margin=1in]{geometry}
\usepackage{booktabs}

\setlength{\parindent}{0pt}
\setlength{\parskip}{1em}

\title{CS 2050 Fall 2025 Homework 9 Solution}
\author{Showmick Das}
\date{November 14th, 2025}

\begin{document}

\maketitle

\section*{Problem 3: Subset Sum Proof}
\textbf{Problem:} Prove that, among all subsets of 4 integers chosen from \(\{0, 1, 2, \dots, 9\}\), there are nine distinct subsets with the same sum.

\textbf{Proof:}
We will use the \textbf{pigeonhole principle}.

\begin{enumerate}
    \item First, calculate the total number of 4-element subsets:
    \[
    \binom{10}{4} = \frac{10!}{4!6!} = \frac{10 \times 9 \times 8 \times 7}{4 \times 3 \times 2 \times 1} = 210
    \]
    
    \item Next, determine the range of possible sums:
    \begin{itemize}
        \item Minimum sum: \(0 + 1 + 2 + 3 = 6\)
        \item Maximum sum: \(6 + 7 + 8 + 9 = 30\)
    \end{itemize}
    Thus, there are \(30 - 6 + 1 = 25\) possible sums.
    
    \item Apply the pigeonhole principle:
    \begin{itemize}
        \item \textbf{Pigeons:} 210 subsets
        \item \textbf{Pigeonholes:} 25 possible sums
    \end{itemize}
    By the pigeonhole principle, at least one sum must be shared by at least:
    \[
    \left\lceil \frac{210}{25} \right\rceil = \left\lceil 8.4 \right\rceil = 9
    \]
    distinct subsets.
\end{enumerate}

Therefore, there exist nine distinct subsets with the same sum.
\[
\boxed{9}
\]

\end{document}